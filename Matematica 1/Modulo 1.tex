\documentclass[a4paper]{article}
\usepackage{amssymb,amsmath}

\begin{document}
\section{Rectas}

\paragraph{Ecuacion general de la recta}
	\begin{equation}
		Ax + By + C = 0
	\end{equation}

\paragraph{Ecuacion explicita de la recta}
	\begin{equation}
		y = mx + b
	\end{equation}

\paragraph{Ecuacion  de la pendiente}
	\begin{equation}
		y ={\frac{y_2 - y_1}{x_2 - x_1}}
	\end{equation}

\paragraph{Ecuacion para sacar cordenadas (x,y) apartir de dos puntos.}
	\begin{equation}
		\frac{y-y_1}{y_2 - y_1} = \frac{x-x_1}{x_2 - x_1}
	\end{equation}
\paragraph{Paralelismo}
	\begin{equation}
 		m = m'  
	\end{equation}
\paragraph{Perpedicularidad}
	\begin{equation}
 		m.m' = -1
	\end{equation}

\paragraph{Ecuacion de la distancia entre dos puntos}
	\begin{equation}
		d=d(P_1,P_2)=\sqrt{(x_2 - x_1)^{2} + (y_2 - y_1)^{2}}
	\end{equation}

\section{Circunferencia}

\paragraph{Ecuacion Estandar}
	\begin{equation}
		(x -\alpha)^{2} + (y - \beta)^{2} = r^{2}
	\end{equation}

\paragraph{Ecuacion} 
	\begin{equation}
		A x^{2}+ B y^{2} + C x^{2} +D y^{2} + E = 0
	\end{equation} 

\section{Parabola}
\begin{description}
	\item Vertice: Es el punto donde la parábola cruza su eje.

	\item C: Distancia del vertice al foco y del vertice a la directriz.

	\item Directriz: Recta fija perperpendicular al eje focal.

\end{description}
 

\paragraph{Ecuaciones}
\begin{description}
	\item Eje focal paralelo al eje x
\end{description}	
	\begin{equation}
		(y - \beta)^{2}= 4c(x - \alpha)
	\end{equation}
\begin{description}
	\item Eje focal paralelo al eje y
\end{description}
		\begin{equation}
		(x - \alpha)^{2}=4c(y - \beta)
	\end{equation}


\section{Conjuntos}
\begin{description}
	\item Definicion: es una coleccion de objetos representada con una letra mayuscula A,B,C...
	\item Se los puede definir de dos maneras:
\end{description}
\begin{itemize}
	\item Explicita o extencion donde escribes todos sus elementos. 
	\item Comprencion donde se enuncia una propiedad  que caracteriza a cada uno de sus elementos.
\end{itemize}



\paragraph{Operaciones entre conjuntos}
\begin{itemize}

\item Union
	\begin{equation}
		A \cup B = \{x; x \in A  \vee x \in  B\}
	\end{equation}

\item Intersecion
	\begin{equation}
		A \cap B = \{x; x \in A  \wedge x \in  B\}
	\end{equation}

\item Diferencia
	\begin{equation}
		A - B = \{x; x \in  A - x \notin B\}
	\end{equation}
\end{itemize}
\paragraph{Reglas de equivalencia}
\begin{itemize}

\item{Asociatividad}
	\begin{equation}
		p \wedge (q \wedge r) \Leftrightarrow (p \wedge q) \wedge r
	\end{equation}
	\begin{equation}
	p \vee (q \vee r) \Leftrightarrow (p \vee q) \vee r
	\end{equation}
\item{Conmutativa}
	\begin{equation}
		p \wedge q \Leftrightarrow q \wedge p
	\end{equation}
	\begin{equation}
		p \vee q \Leftrightarrow q \vee p
	\end{equation}
\item{Distributiva}
	\begin{equation}
		p \wedge (q \vee r) \Leftrightarrow (p \wedge q) \vee (p \wedge r)
	\end{equation}
	\begin{equation}
		p \vee (q \wedge r) \Leftrightarrow (p \vee q) \wedge (p \vee r)
	\end{equation}
\item{De Morgan}
	\begin{equation}
		\neg(p \wedge q) \Leftrightarrow ( \neg p \vee \neg q)
	\end{equation}
	\begin{equation}
		\neg(p \vee q) \Leftrightarrow ( \neg p \wedge \neg q)
	\end{equation}
\item{Implicacion}
	\begin{equation}
 		(p \rightarrow q) \Leftrightarrow (\neg p \vee q) \Leftrightarrow \neg(p \wedge \neg q)
	\end{equation}
\item{Contraposicion}
	\begin{equation}
		(p \rightarrow q) \Leftrightarrow (\neg q \rightarrow \neg p)
	\end{equation}
\end{itemize}
\paragraph{Reglas de Inferencia}
\begin{itemize}
\item{Simplificacion}
	\begin{equation}
		(p \wedge q) \Rightarrow p  
	\end{equation}
	\begin{equation}
		(p \wedge q) \Rightarrow q
	\end{equation}

\item{Adicion}
	\begin{equation}
		p \Rightarrow (p \vee q)
	\end{equation}
	\begin{equation}
		q \Rightarrow ( q \vee p)
	\end{equation}

\end{itemize}


\section{Funciones}
\paragraph{Inyectiva}
\begin{itemize}
	\item Una funcion F: $A \rightarrow B$ es inyectiva si $\forall$ $x_1,x_2 \in$ A = Dom F, $x_1 \neq x_2$ y sus imagenes $F(x_1)$ ,$F(x_2)$ son distintas $\forall$ $x_1$,$ x_2 \in$ Dom F.

\item Si $x_1 \neq x_2 \Longrightarrow F(x_1) \neq F(x_2)$.

\item Pueden quedar elementos sueltos en el codominio.
\end{itemize}

\paragraph{Suryectiva}
\begin{itemize}
	\item Una funcion F: $A \rightarrow B$ es suryectiva si $\forall$y$\in B$ $\exists$ algun $x \in A$ talque $F(x)= y$
	\item Img F : B
	\item No deben quedar elementos sueltos.	
\end{itemize}

\paragraph{Biyectiva}
\begin{itemize}
	\item  Cuando una funcion F: $A \rightarrow B$ es inyectiva y suryectiva entonces es biyectiva. 
\end{itemize}

\paragraph{Regla general}
\begin{description}
	\item Si una funcion tiene su exponente par entonces no es "inyectiva".
\end{description}

\section{Sucesiones}

\paragraph{Aritmetica}
\begin{description}
	\item Forma explicita : $a_n = a_1 + (n-1).d$
	\item Forma por recurrencia :	\begin{equation*}
	\begin{cases}
    	     a_1 = n \\ 
    	     a_n = a_{n-1} + d                
    \end{cases}
\end{equation*}
\end{description}
	La suma de los primeros terminos de una sucesion aritmetica  $a_1$,$a_2$,$a_3$...$a_n$ es
	\begin{equation}
		S_n = \sum_{i=1}^{n}a_i =\frac{n(a_1 + a_n)}{2} 
	\end{equation}	

\paragraph{Geometrica}
\begin{description}
	\item Forma explicita : $a_n = a_1 . r^n$
	\item Forma por recurrencia : \begin{equation*}
	\begin{cases}
    	    a_1 = n \\ 
    	    a_n = a_{n-1} . r	                  
    \end{cases}
\end{equation*}
\end{description}
	La suma de los primeros terminos de una sucesion geometrica  $a_1$,$a_2$,$a_3$...$a_n$ es:
	\begin{equation}
		S_n = \sum_{i=1}^{n}a_i =\frac{a_1(1 - r^n)}{1-r} 
	\end{equation}

\section{Induccion}
\begin{description}
	\item Si $\forall$n $\in (\Bbb{N})$,$P(n)$
\end{description}
	
\begin{enumerate}
	\item Probar P(0) o P(1);		
	\item Hipotesis Inductiva...Suponer Verdadero  un P(k);
	\item Probar P(k + 1);
\end{enumerate}

\begin{description}
\item Si se cumple todo lo de arriba entonces por principio de induccion completa(PIC) $P(n)$ vale $\forall$n $\geqslant$ 0  o 1.
\end{description}


\end{document}

