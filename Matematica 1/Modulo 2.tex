\documentclass[a4paper]{article}
\usepackage{amssymb,amsmath}

\begin{document}
\section{Factorial-Permutacion-Conbinatoria}

\paragraph{Factorial}
	\begin{itemize}
		\item $0!$ = 1		 
		\item $1!$ = 1;
		\item  $\forall n \geq 1; n(n-1)(n-2)....1;$ = $n!$ 
	\end{itemize}

\paragraph{Principio de multiplicacion}

\begin{itemize}
\item Si una actividad de t pasos sucesivos y el paso 1 puede realizarce de $n_1$ formas, el paso 2 puede realizarse de $n_2$ formas ,...,el paso t puede realizarse de $n_t$ formas,	entonces el numero de diferentes resultados posibles es	$n_1$,$n_2$...$n_t$.
\end{itemize}

\paragraph{Principo de la suma}
\begin{itemize}
	\item si una esperiencia $E_1$  se puede efectuar de $n_1$ maneras $\neq$ y una experiencia $E_2$ se puede efectuar de $m_2$ maneras y no se efectuan simultaneamente $E_1$ y $E_2$, el numero total de maneras de efectuar  $E_1$ o $E_2$ es $m_1 + m_2$.
\end{itemize}

\paragraph{Permutacion}
\begin{itemize}
	\item P(n) = n! de n elementos distintos.
\end{itemize}

\paragraph{R-permutaciones}
\begin{itemize}
	\item Al numero de r-peromutaciones de un conjunto de n elementos distintos lo denotaremos P(n,r):
	\begin{equation*}
		P(n,r) = n(n-1)(n-2)...(n-r+1)
	\end{equation*}

	\item Sean n y r numeros naturales tales que 1$\leq$ r $\leq$ n:
	\begin{equation*}
		P(n,r)= \dfrac{n!}{(n-r)!}
	\end{equation*}

	\item Una sucecion S de n objetos entre los que hay $n_1$ elementos iguales del tipo 1, $n_2$ elementos iguales del tipo 2.... $n_k$ elementos iguales del tipo K, el numero de permutaciones distinguibles de los elementos de S es:
	\begin{equation*}
		\dfrac{n!}{n_1!n_2!...n_k!}
	\end{equation*}
\end{itemize}

\paragraph{Combinaciones}
\begin{itemize}
	\item Dado un conjunto de n elementos dedistintos los cuales no estan ordenados n: objetos K$\leq$n
	\begin{equation}
		C(n,k) = \binom n k  = \dfrac{n!}{k!(n-k)!}
	\end{equation}
\end{itemize}

\section{Matrices}

\paragraph{Matriz mxn}

\begin{itemize}
	\item m = filas, n = columnas, donde $a_{ij}$ es la posicion actual.
\end{itemize}
\begin{equation}
\left[ 
\begin{array}{ccccc}
	a_{11} & a_{12} & a_{13} & ... & a_{1n} \\
	a_{21} & a_{22} & a_{23} & ... & a_{2n} \\
	... & ... & ... & ... & ... \\
	a_{m1} & a_{m2} & a_{m3} & ... & a_{mn}
\end{array}
\right] 
\end{equation}
 
\paragraph{Matriz Nula}
\begin{itemize}
	\item Todos sus coeficientes son nulos, puede ser cuadrada o rectangular: 
	\begin{equation}
	(O)_{ij} = 0, \forall i , \forall j
	\end{equation}
\end{itemize}

\paragraph{Matriz identidad I}
\begin{itemize}
\item matriz diagonal, en la que los coeficientes de la diagonal principal son todos 1.

\begin{equation}
	I : \begin{cases}
			(I)_{ij} = 1,  i = j \\
			(I)_{ij} = 0,  i \neq j		
		\end{cases}
\end{equation}
\end{itemize}

\paragraph{Suma de matrices}
\begin{itemize}
	\item Sean A,B matrices mxn, la suma de A + B es la matriz mxn donde cada coeficiente $(A + B)_{ij}$ = $a_{ij}$ + $b_{ij}$ = $(A)_{ij}$ + $(B)_{ij}$.
\end{itemize}

\paragraph{propiedades}
\begin{itemize}
	\item Sean A,B,C $\in$ $R^{mxn}$.
\end{itemize}

\begin{enumerate}
	\item Asociatividad : A +(B + C) = (A + B) + C
	\item Existencia del neutro: (matriz nula O) $\in$ $R^{mxn}$ talque $\forall$ matriz A $\in$ $R^{mxn}$, A + O = O + A = A;
	\item $\forall$ matriz A $\in$ $R^{mxn}$ $\exists$ una matriz B  $\in$ $R^{mxn}$ talque A + B = B + A = 0, siendo B la opuesta de A y se indica -A.
	\item Conmutatividad:  $\forall$ par de matrices A, B $\in$ $R^{mxn}$, A + B = B + A
\end{enumerate}

\paragraph{Producto de un escalar (numero real) por una matriz}

\begin{itemize}
 \item sean A $\in$ $R^{mxn}$ y $\alpha$ $\in$ R, $\alpha$.A es la matriz de $R^{mxn}$ talque cada coeficiente $(\alpha.A)_{ij}$ = $\alpha$.$(A)_{ij}$ = $\alpha.a_{ij}$  
\end{itemize}	

\paragraph{propiedades}
\begin{itemize}
	\item Sean $\alpha$, $\beta$ escalares y A,B matrices $mxn$.
\end{itemize}

\begin{enumerate}
	\item $\alpha$.(A + B) = $\alpha$.A + $\alpha$.B
	\item ($\alpha$ + $\beta$).A = $\alpha$.A + $\beta$.A
	\item ($\alpha$ $\beta$).A = $\alpha$.($\beta$.A)
	\item I.A = A
\end{enumerate}

\paragraph{Producto de matrices}

\paragraph{propiedades}
\begin{enumerate}
	\item asosiatividad (A.B).C = A.(B.C)
	\item la matriz identidad I (nxn) cumple : A.I = I.A = A, esta conmuta con cualquier matriz del mismo orden.
	\item distributividad del product en la suma A.(B+C) = A.B + A.C 
	\item la conmutatividad NO es una propiedad general del producto de matrices entonces A.B = 0 NO implica A = 0 $\vee$ B = 0
	\item la existencia del inverso tampoco es una propiedad general de las matrices, si una matriz no es cuadrada no posee inversa.
\end{enumerate}

\paragraph{Matriz Inversa}
\begin{itemize}
	\item sea A $\in$ $R^{mxn}$, A tiene inversa si existe una matriz B $\in$ $R^{mxn}$ talque A.B = I $\wedge$ B.A = I, y se piden las dos condiciones porque el producto de matrices no posee emn general la propiedad conmutatividad.	
\end{itemize}

\paragraph{propiedad}
\begin{enumerate}
	\item si A,B matrices nxn inversibles entonces $(A.B)^{-1}$= $B^{-1}$.$A^{-1}$
\end{enumerate}

\paragraph{Matriz Traspuesta}
\begin{itemize}
	\item si A es mxn, su traspuesta $A^T$ es nxm y tiene coeficiente $(A^T)_{ij}$=$(A)_{ij}$
\end{itemize}

\begin{enumerate}
	\item $(A^T)^T$ = A
	\item $(A + B)^T$ = $A^T$ + $B^T$
	\item K escalar de R, entonces $(K.A)^T$ = K.$A^T$
	\item si A es nxn y B es nxn, entonces $(A.B)^T$ = $B^T$.$A^T$
	
	
	\item $\textbf{Matriz simetrica}$ matriz A cuadrada nxn talque A = $A^T$.
	\item $\textbf{Matriz antisimetrica}$ matriz A cuadrada nxn talque $A^T$ = -A.
\end{enumerate}

\paragraph{Matriz Escalonada}
\begin{itemize}
	\item el numero de ceros que preceden al primer coeficiente no nulo, aumenta en las filas siguientes. El primer coeficiente no nulo de cada fila es el coeficiente princical.
\end{itemize}
		
\paragraph{Matriz Reducida}

\begin{itemize}
	\item una matriz no nula A es $\textbf{reducida}$ si es escalonada y ademas sus coeficientes principales cumplen :
	\begin{enumerate}
		\item son los unicos coeficientes distintos de cero en sus respectivas columnas.
		\item son todos iguales a 1.
	\end{enumerate}
\end{itemize}

\paragraph{Operaciones elementales sobre las filas de una matriz}
\begin{enumerate}
	\item multiplicar una fila por un escalar $\alpha.F_K$ $\longrightarrow$ $F_K$.
	\item sumar a la fila $F_h$ la fila $F_K$ multiplicada por $\alpha$ no nulo
	 $F_h$+$\alpha$.$F_K$ $\longrightarrow$ $F_h$. 
	\item permutar dos filas $F_h$ $\longleftrightarrow$ $F_K$.
	\item se llama rango de una matriz A,indicado con r(A), al numero de filas no nulas de la escalonada o la reducida $R_A$ equivalente con A.
\end{enumerate}

\paragraph{Calculo de la inversa de una matriz}
\begin{itemize}
	\item dada una matriz cuadrada A nxn, se amplia ubicando a su derecha la matriz identidad I del mismo orden de A, formando la matriz(A I) de orden nx2n.
	\item a esta matriz ampliada (A I) se le aplica operaciones elementales y si $R_A$ es la identidad entonces donde se ubico I se tendra la inversa $A^{-1}$ de A.
\end{itemize}

\paragraph{Teorema Rouchè-Frobenius}
\begin{itemize}
	\item Un sistema A.x = b es compatible $\Leftrightarrow$ r(A) = r(A$\mid$b').
\end{itemize}
\begin{enumerate}
	\item r(A) = r(A $\mid$ b') = \textbf{n} incognitas compatible con solucion unica DETERMINADA.
	\item r(A) = r(A$\mid$b') $<$ \textbf{n} incognitas compatible con infinitas soluciones INDETERMINADO.
	\item r(A) $<$ r(A$\mid$b') sistema INCOMPATIBLE.
\end{enumerate}

\paragraph{Sistemas homogeneos de ecuaciones lineales}
\begin{itemize}
	\item un sistema de ecuacion lineal es homogeneo si todos los terminos independientes son ceros A.x = 0
	\item todo sistema homogeneo es compatible: solucion trivial todas las incognitas se remplazan por ceros.
	\item si el sistema es determinado, tiene una unica solucion que debe necesariamente ser la trivial.
	\item si el sistema es indeterminado tendra ademas de la trivial otras infinitas soluciones.
\end{itemize}

\paragraph{Determinantes de una matriz}
\begin{itemize}
	\item estan definidas en las matrices cuadradas, A de 2x2 
\end{itemize}
\begin{enumerate}
	\item se calcula $\mid A \mid$ = $a_{11}.a_{22} - a_{21}.a_{12}$.
\end{enumerate}

\paragraph{propiedades}
\begin{itemize}
	\item \textbf{A} cualquiera nxn.
\end{itemize}

\begin{enumerate}
	\item det$(A^T)$ = det A.
	\begin{itemize}
		\item \textbf{corolario} toda prop de los determinantes que se enuncia para las filas vale tambien para las columnas.
	\end{itemize}
	\item si se multiplican todos los coeficientes de una fila de A por un escalar \textit{c} entonces el detA queda multiplicado por \textit{c}.
	\begin{itemize}
		\item \textbf{corolario} si concideramos la matriz \textit{c}.A(producto de un escalar \textit{c} por la matriz A ) entonces det(\textit{c}.A) = $c^n$.det A
	\end{itemize}
	\item Si una fila (o columna) de A tiene todos sus coeficientes iguales a 0 entonces detA = 0
	\item sean A, B matrices $R^{nxn}$ entonces det(A.B) = detA.detB
	\begin{itemize}
		\item para la matriz I nxn, detI = 1, cualquiera sea n.
	\end{itemize}
	\item A tiene inversa si y solo si det(A) es distinto de 0.
	\item Si la matriz A tiene inversa $A^{-1}$, entonces det($A^{-1}$) = $[det A]^{-1}$ = $\dfrac{1}{det A}$.
	\item A.x = b tiene solucion unica (es compatible determinado) si y solo si la matriz A tiene inversa.
\end{enumerate}

\paragraph{Espacios vectoriales}
\begin{itemize}
	\item un espacio vectorial $\textbf{V}$ es un conjunto no vacio, cuyos elementos se pueden sumar entre si y multiplicar por escalares simpre perteneciendo a $\textbf{V}$ incluso su producto.
\end{itemize}
\paragraph{Propiedades}
\begin{itemize}
	\item $\vec{u},\vec{x},\vec{w},\vec{v}$ $\in$ $\textbf{V}$.
\end{itemize}

\begin{enumerate}
	\item $(\vec{u}$ + $\vec{v})$ + $\vec{w}$  = $\vec{u}$ + ($\vec{v}$ + $\vec{w})$ = $\vec{u}$ + $\vec{v}$ + $\vec{w}$.
	\item neutro, vector nulo $\vec{0}$; $\vec{v}$ + $\vec{0}$ = $\vec{v}$ = $\vec{0}$ + $\vec{v}$.
	\item todo $\vec{v}$ tiene opuesto -$\vec{v}$ talque  $\vec{v}$ + (-$\vec{v}$) = $\vec{v}$ - $\vec{v}$ = $\vec{0}$.
	\item $\forall$ $\vec{v}$,$\vec{u}$;   $\vec{u}$ + $\vec{v}$ = $\vec{v}$ + $\vec{u}$.
\end{enumerate}
\begin{itemize}
	\item $\alpha$,$\beta$ $\in$ $\Bbb{R}$.
\end{itemize}
\begin{enumerate}
	\item ($\alpha$ + $\beta$).$\vec{v}$ = $\vec{v}$ + $\alpha$. + $\vec{v}$ $\beta$.
	\item $\alpha$.($\vec{u}$ + $\vec{v}$) = $\alpha$.$\vec{u}$ + $\alpha$.$\vec{v}$.
	\item ($\alpha$.$\beta$).$\vec{v}$ = $\alpha$.($\beta$.$\vec{v}$).
	\item 1 .$\vec{v}$ = $\vec{v}$.	  

\end{enumerate}
	

\paragraph{Subespacios vectoriales}
\begin{itemize}
	\item un conjunto $\textbf{S}$ incluido en un espacio vectorial $\textbf{V}$ es un subespacio de $\textbf{V}$ si satisface las siguientes condiciones:
\end{itemize}
\begin{enumerate}
	\item el vector nulo 0 $\in$ $\textbf{S}$.
	\item Si $\vec{v}$,$\vec{w}$ $\in$ $\textbf{S}$ entonces $\vec{v}$+$\vec{w}$ tambien $\in$ $\textbf{S}$.
	\item $\alpha$ $\in$ $\Bbb{R}$ $\longrightarrow$ $\alpha$.$\vec{v}$ $\in$ $\textbf{S}$. 
\end{enumerate}

\paragraph{Conbinacion lineal.Vectores Generadores}
\begin{itemize}
	\item Un vector $\textbf{V}$ es una convinacion lineal de los vectores $\vec{v}_1$,$\vec{v}_2$,...,$\vec{v}_k$ $\in$ $\textbf{V}$ si existen escalares $c_1,c_2,...,c_k$ no necesariamente distintos tales que:
	\begin{equation}
		v = c_1.\vec{v}_1,c_2.\vec{v}_2,...,c_k\vec{v}_k = \sum_{j=1}^{k} c_j.\vec{v}_j
	\end{equation}
	\item Cuando un vector es convinacion lineal de otros, esta no es necesariamente unica.Dado un conjunto de vectores no siempre uno de ellos es una convinacion lineal de los otros.
\end{itemize}
\paragraph{Subespacio generado y generadores}
\begin{itemize}
	\item si $\textbf{v}$ = $\vec{v}_1$,$\vec{v}_2$,...,$\vec{v}_k \in \textbf{V}$.  
	\item $\vec{v}$ es convinacion lineal de $\vec{v}_1$,$\vec{v}_2$,...,$\vec{v}_k$ :	
	\begin{equation}
		\vec{v} = \alpha_1.\vec{v}_1+\alpha_2.\vec{v}_2,...,\alpha_k.\vec{v}_k =  \sum_{j=1}^{k} \alpha_j.\vec{v}_j
	\end{equation}
			
	\begin{equation}
	 v = \sum_{j=1}^{k} c_j.v_j = \langle\vec{v}_1,\vec{v}_2,...,\vec{v}_k\rangle
	\end{equation}	 
	
	\item subespacios generados por $\vec{v}_1$,$\vec{v}_2$,...,$\vec{v}_k$
	\begin{equation}
		\textbf{S} = \lbrace \alpha_1.\vec{u}+\alpha_2.\vec{v};\alpha_1,\alpha_2 \in \Bbb{R}\rbrace = \langle \vec{u},\vec{v}\rangle
	\end{equation}
\end{itemize}

\paragraph{Dependencia e Independencia lineal de vectores}
\begin{itemize}
	\item un conjunto de vectores $\vec{v}_1$,$\vec{v}_2$,...,$\vec{v}_k \in \textbf{V}$ es \textbf{linealmente dependiente} si se verifica una de las dos condiciones siguientes:
\end{itemize}
\begin{enumerate}
	\item alguno de los vectores $\vec{v}_1$,$\vec{v}_2$,...,$\vec{v}_k$ es una convinacion lineal de los demas.
	\item existen escalares $c_1,c_2,...,c_k$ que no son todos nulos tales que 
	\begin{equation}
		 c_1.\vec{v}_1,c_2.\vec{v}_2,...,c_k\vec{v}_k = \sum_{j=1}^{k} c_j.v_j = 0
	\end{equation}
\end{enumerate}
\begin{itemize}
	\item los vectores $\vec{v}_1$,$\vec{v}_2$,...,$\vec{v}_k$  de un espacio V son \textbf{linealmente independientes} si la realcion de arriba cumple tambien que $\vec{v}_1 = 0$,$\vec{v}_2 = 0$,...,$\vec{v}_k = 0$. 
	\end{itemize}
	
\paragraph{Bases y dimenciones}
\begin{itemize}
	\item sean $\vec{v}_1,\vec{v}_2,...,\vec{v}_n$ vectores no nulos de un espacio vectorial \textbf{V}.El conjunto $\lbrace\vec{v}_1,\vec{v}_2,...,\vec{v}_n\rbrace$ es una base de \textbf{V} y a la vez se dice que V tiene dimension n, si cumple:
\begin{enumerate}
	\item $\vec{v}_1,\vec{v}_2,...,\vec{v}_n$ son linealmente independientes.
	\item $\vec{v}_1,\vec{v}_2,...,\vec{v}_n$ generan $\textbf{V}$.
\end{enumerate}	
\end{itemize}

\paragraph{Propiedades}
\begin{itemize}
	\item sea \textbf{V} un espacio vectorial de dimension \textbf{n}, entonces: 
\end{itemize}
\begin{enumerate}
	\item la base de \textbf{V} no es unica.
	\item todas las bases de \textbf{V} tienen exactamente \textbf{n} elementos.
	\item cualquier subconjunto de \textbf{V} que contenga $n+1$ vectores es linealmente dependiente.
	\item si un conjunto de \textbf{V} tiene \textbf{n} vectores linealmente independientes, entonces es una base de \textbf{V}.
\end{enumerate}

\end{document}
