\documentclass[a4paper, twocolumn]{article}
\usepackage[utf8]{inputenc}
\usepackage{amssymb,amsmath}
\usepackage{multicol}
\setlength{\columnsep}{1cm}
\usepackage{lscape}

\begin{document}
\begin{landscape}
\begin{multicols}{2}
\begin{multicols}{2}
[
\section*{Manejo de Archivos}
]
\begin{itemize}
	\item C-x C-f : abre un archivo
	\item C-x C-s : guaradr archivo actual, sin preguntar
	\item C-x s : guardar todos los acrivos, preguntar
	\item C-x C-w : guardar cambiando el nombre

	\item M-x recover-file : recupera un archivo autoguardado
	\item C-x C-c : salir

\end{itemize}


\end{multicols}
\begin{flushleft}
\begin{itemize}
	\item Al guardar un archivo, emacs guarda un buckup(copia de respaldo) de la vercion anterior, con el nombre $archivo\backsim$.
	\item Cada cierto tiempo se guarda automaticamente una copia con el nombre \#archivo\#
\end{itemize}
\end{flushleft}

\section*{Cortar/copiar/pegar}
\begin{multicols}{2}
\begin{itemize}
	\item C-k: cortar desde el cursor hatsa el final de la linea
	\item C-y : pega donde esta el cursor
	\item C-space : inicia selecion
	\item M-w : copia la region seleccionada
	\item C-w : corta la region selecionada
	\item C-g : cancelar, terminar seleccion
\end{itemize}
\end{multicols}

\section*{Comandos de Cursor}
\begin{multicols}{2}
\begin{itemize}
	\item C-a : ir al principio de linea
	\item C-e : ir al final de linea
	\item M-$<$ : ir al principio del buffer
	\item M-$>$ : ir al final del buffer
	
	\item C-$\diagup$ : deshacer
	\item C-g C-$\diagup$ : rehacer
\end{itemize}
\end{multicols}

\section*{Administrador de buffers}
\begin{multicols}{2}
\begin{itemize}
	\item C-x b : cambiar de buffer
	\item C-x C-b : cambiar de buffer mostrando una lista en un panel nuevo
	\item C-x o : cambia de ventana (panel)
	
	\item C-x 1 : deja la ventana actual, cierra las demas
	\item C-x 2 : divide verticalemnte el panel (arriba-abajo)
	\item C-x 3 : divide horizntalmente el panel (izq-der)
	\item C-x 0 : cierra la ventana actual
\end{itemize}
\end{multicols}


\section*{Comandos de busqueda}
\begin{multicols}{2}
\begin{itemize}
	\item C-s : busqueda dinamica
	\begin{itemize}
		\item C-j : nueva linea
		\item C-g : cancelar
		\item C-s : moverse entre ocurrecioas
	\end{itemize}
	\item M-C-s : busqueda mediante regexp(expresion regular)
	\item M-\% : buscar y remplazar
	\begin{itemize}
		\item y : aceptar
		\item n : omitir
		\item ! : todos		
		\item q o Enter : finalizar
	\end{itemize}
	\item M-s o : buscar mostrando ocurrencias en otra ventana (Enter para saltar)
	\item M-x grep : busca mediante regexp en archivos (Enter archivo deseado)
	\item M-x rgrep : busca como grep pero recursivamente en directorio
\end{itemize}
\end{multicols}

\section*{Comandos}
\begin{flushleft}
Emacs entiende elisp, su propio dialecto de lisp
	\begin{flushleft}
		M-x comando : ejecuta el comando y debes presionar TAB para autocompletar.
		ejemplos =
		\begin{itemize}
		    \item M-x pwd : imprime el directorio actual
			\item M-x indent-region : indenta una region seleccionada
			\item C-x C-e : evalua la expresion lisp detras del cursor y si es variable muestra su contenido
		\end{itemize}
			
\end{flushleft} 
\end{flushleft}


\section*{Configuracion y gestion de paquetes}

\begin{flushleft}
\begin{itemize}
	\item La configuracion de emacs se guarda en $\thicksim/.emacs.d/$
	\item se edita un archivo $.el$ con las sentencias lisp
	\item se ejecuta al iniciar emacs
\end{itemize}

	\item M-x list-packages : muestra la lista de paquetes
	\begin{itemize}
		\item i : seleccionar para instalar
		\item d : seleccionar para quitar
		\item luego aplicar cambios con $x$
	\end{itemize}
\end{flushleft}


\end{multicols}
\end{landscape}
\end{document}